
\chapter{What's next?}

\section{Is Bitcoin de MySpace van crypto?}

Waarom schrijf ik een boek over Bitcoin en niet over het grotere crypto ecosysteem? Zijn er geen duizenden andere munten? Wat maakt bitcoin zo speciaal, anders dan dat het de eerste decentrale cryptovaluta is? Is het niet trager dan andere munten en met minder mogelijkheden dan de nieuwe concurrenten?

Dit is een veelgestelde vraag van nieuwkomers. Na een eerste introductie tot bitcoin en haar werking, is de volgende logische vraag vaak: "Zo'n blockchain is interessant, maar hoe weten we dat er geen betere versie opdaagt die van bitcoin de MySpace maakt van crypto?"

Bitcoins concurrentievoordeel is veel, veel groter dan het voordeel destijds van MySpace. Laten we eens kijken wat een concurrent nodig heeft om bitcoin te vervangen.

\section{Gemakkelijker verkoopbaar en grotere liquiditeit}

Het eerste om te begrijpen is dat een vergelijking met MySpace en Facebook zwak is, omdat het je niets kost om tegelijkertijd op beide platforms een account te onderhouden. Dit is precies wat er gebeurde tijdens de transitie van de één naar de ander. Zodra genoeg kritieke massa was overgestapt naar Facebook, stopte men met MySpace.

Maar dit is niet hoe geld werkt. Als jij een euro aan bitcoin hebt, is dat een euro die je niet in een andere munt hebt. Je moet er bewust voor kiezen om de één voor de ander te verhandelen. Het is onmogelijk om tegelijkertijd hetzelfde vermogen op te slaan in verschillende munten. Vraag jezelf eens af: waarom zou je geld bewaren in iets anders dan de meest liquide en best geaccepteerde valuta. Het enige antwoord is speculatie. Als je niet in staat bent om de gehele economie te overtuigen om de andere munt te accepteren, dan is er geen enkele manier dat het dominant wordt. 

De liquiditeit van bitcoin is vele malen groter dan alle concurrenten. Op moment van schrijven is de marktkapitalisatie van bitcoin ongeveer \$1 biljard (1000 miljard).\footnote{\href{https://messari.io/onchainfx}{messari.io/onchainfx}} De eerstvolgende concurrent, Ethereum, heeft een kapitalisatie van slechts \$500 miljard. Dit zegt nog niets over de ware liquiditeit, de mate waarin je een aanzienlijk bedrag kunt verkopen zonder dat de prijs drastisch zal dalen. 

Liquiditeit heeft een sneeuwbaleffect. Het meeste liquide geld aanhouden, betekent dat andere mensen het willen, waardoor de liquiditeit nog verder toeneemt. Door vast te houden aan alles behalve het meest liquide geld, straf je jezelf in de hoop dat anderen hetzelfde gaan doen. Er is simpelweg geen enkele economisch motief om van de één op de andere dag massaal op een concurrent over te stappen.

\section{Aantoonbaar tien jaar \$1000 miljard beveiligd}

Bitcoin begon in 2009 als internet experiment voor computernerds. Slechts 1 jaar later volgde de eerste transactie (een pizza voor 10000 bitcoins) en inmiddels is 1 bitcoin al meer dan \$60.000 waard. Dit gebeurde in relatieve rust, zonder al te veel ophef. Door de jarenlange aanvallen is immuunsysteem van bitcoin inmiddels van wereldklasse, met het grootste netwerk aan rekenkracht ter wereld. Al tien jaar lang blijkt het onmogelijk om te hacken en het beveiligt inmiddels meer dan 1000 miljard dollar.

Het is haast onmogelijk om vandaag de dag in stilte een nieuwe cryptovaluta te lanceren. Laten we eens kijken naar een alternatieve blockchain, EOS, met een waarde van ongeveer ~\$10 miljard bij de lancering van het netwerk (en vandaag nog slechts de helft waard). Het netwerk moest slechts 2 dagen na aanvang al op slot wegens fouten in de code. Deze fouten werden zonder al te veel toezicht of review verholpen. Durf jij hier \$1000 miljard in bewaren? Misschien bestaat EOS over 10 jaar nog, wie het weet mag het zeggen. Maar tegen die tijd is bitcoin 20 jaar oud en beveiligt het biljarden aan waarde.  

\section{Aanvallen met bestaande rekenkracht afslaan}

Met duizenden crypto's en allerlei hashing-algoritmes, staat iedere nieuwe munt onder bedreiging van een 51\%-aanval met bestaande rekenkracht. Bitcoin Gold en vele andere munten zijn hier al slachtoffer van geweest.

Een nieuwe concurrent moet deze aanvallen van bestaande rekenkracht dus kunnen overleven, of moet op zoek naar een algoritme zonder gespecialiseerde ASIC's. Maar zonder ASIC's is het systeem juist weer kwetsbaar voor een aanval met standaard GPU's. Een ander probleem is dat een nieuwe concurrent niet zomaar vanaf dag 1 veel waarde kan beveiligen, zoals EOS probeerde. Dit is roekeloos en voorbode voor een gecentraliseerd systeem. Dat betekent dus ook dat een nieuwe concurrent niet zomaar geld op kan halen, maar net zoals bitcoin langzaam zal moeten groeien om de beveiliging proportioneel te laten groeien. Echter, met langzame groei wordt het haast onmogelijk om bitcoin nog in te halen. 

\section{Decentralisatie}

Een groot gedeelte van bitcoin's beveiligingsmodel berust op een hoge graad van decentralisatie. Dit betekent dat het moeilijk is om het protocol te wijzigen en men erop kan vertrouwen dat het de eigenschappen zoals beloofd in de broncode zal waarborgen (gelimiteerd aanbod, etc). Bitcoin bewees deze eigenschap te bezitten toen een groot aantal bedrijven en miners de blockgrootte wilden wijzigen, om het protocol een bepaalde richting op te sturen.\footnote{Lees hier meer over de achterkamertjespolitiek die plaatsvond bij de zogeheten Segwit2X fork:  \href{https://bitcoinmagazine.com/articles/now-segwit2x-hard-fork-has-really-failed-activate}{bitcoinmagazine.com/articles/now-segwit2x-hard-fork-has-really-failed-activate}}. De voorgestelde aanpassingen faalden spectaculair nadat ze werden afgewezen door de gebruikers van bitcoin. 

Een concurrent die decentralisatie nastreeft zal bedrijven en teams met bekende mensen moeten uitsluiten, om de onderdrukking en \textquotedbl{}single points of failure\textquotedbl{} te voorkomen. Ook munten met het motto \textquotedbl{}move fast and break things\textquotedbl{} zijn uitgesloten, want dat kan alleen bij voldoende centralisatie. Al met al is een concurrent dus snel en kwetsbaar voor centralisatie, of traag en niet in staat om bitcoin in te halen.

\section{Trek de beste ontwikkelaars aan}

Zoals de wervelwind van activiteit bij Linux, andere UNIX-achtige besturingssystemen weerhield van concurrentie, zo kan het ook gebeuren bij bitcoin. Elke dag wordt de gemeenschap groter en worden nieuwe bedrijven, met nieuwe diensten, gebouwd bovenop bitcoin. Een concurrent moet ontwikkelaars zien te stelen van een exponentieel groeiend netwerk, met inmiddels honderden bedrijven en tal van educatieve programma's en conferenties.  

\section{Een wereldwijd financieel netwerk}

Bitcoin wordt inmiddels gedragen door een wereldwijd netwerk van handelsbeurzen, futures en andere financiële derivaten bij grote spelers zoals de Chicago Mercantile Exchange (CME), honderden hefboomfonsen en handelskantoren, en een netwerk van mensen die bitcoin al gebruiken als alternatief voor gebroken valuta zoals de Venezulaanse bolivar. Bitcoin laat zich dus niet zo makkelijk vervangen.

Instellingen zoals de CME nemen een nieuwe concurrent pas op als er al voldoende handelsvolume is. Je zult bedrijven er dus van moeten overtuigen om de nieuwe concurrent te accepteren in plaat van bitcoin. Een concurrent die waarschijnlijk minder veilig en minder liquide is, minder competente ontwikkelaars heeft en per definitie over minder wereldwijde adoptie beschikt. 

\section{Solide geld}

Bitcoin is nooit bedoeld als snelle en goedkope manier van betalen. Dat is een groot misverstand. De fundamentele eigenschappen waarbij het grootboek wereldwijd wordt gerepliceerd, staan dat simpelweg niet toe. Daarentegen groeit bitcoin's primaire en reeds bewezen toepassing als censuur-resistent, solide geld. 

Alles wat daarbij komt, zoals het goedkoper maken van internationale overboekingen, zijn kersen op de taart. De meeste concurrenten richten zich nog steeds op snelle betalingen, terwijl dat probleem al redelijk goed is opgelost door vele gecentraliseerde bedrijven. Daarnaast heeft bitcoin daar ook inmiddels een oplossing voor gevonden met het snel groeiende Lightning Netwerk. 

Om te kunnen concurreren op het front van solide geld moeten onveranderbare eigenschappen en decentralisatie ten alle tijde voorop staan bij de ontwikkeling. Bitcoin's ecosysteem is gebouwd door \textit{cypherpunks} en heeft de kans gehad om langzaam te groeien, maar de meeste munten worden ontwikkeld door winstgedreven, gecentraliseerde teams en maken dus weinig kans op slagen.   

\newpage
\section{Toekomstige ontwikkelingen in Bitcoin}

We hebben inmiddels het volledige protocol onder de loep genomen. Laten we eens kijken naar de toekomst en sommige van de aankomende verbeteringen bespreken.

Bitcoin is programmeerbaar geld waar we allerlei diensten bovenop kunnen bouwen. Dit is een volledig nieuw concept en we staan pas aan het begin van wat mogelijk is.

\section{Lightning Netwerk}

Bitcoin heeft verschillende periodes van hoge transactievergoedingen gekend op momenten dat er hoge vraag was naar de blokruimte. Bitcoin is op dit moment slechts in staat om ongeveer 3 tot 7 transacties per seconde te verwerken, op basis van de hoeveelheid transacties die in een blok passen. Ook al kan een batch-transactie honderden mensen betalen in een transactie, dat is nog steeds te weinig capaciteit voor een wereldwijd betalingsnetwerk.    

Het vergroten van de blokruimte is een naïve oplossing en desondanks door veel concurrenten, waaronder Bitcoin Cash, toegepast. Bitcoin heeft gekozen voor een andere route omdat het vergroten van de blokken de decentrale eigenschappen zoals het aantal nodes en de geografische verspreiding van nodes negatief beïnvloedt. 

Een verhoging van de blokgrootte kan er hoe dan ook niet voor zorgen dat bitcoin geschikt wordt als wereldwijd betalingsnetwerk --- het schaalt simpelweg niet genoeg. Welkom bij het Lightning Netwerk: een protocol en verschillende software implementaties om \textit{off-chain} bitcointransacties te verwerken die periodiek kunnen verrekend worden op de blockchain. Over het Lightning Netwerk zouden we een heel boek kunnen schrijven, maar we beperken ons hier tot de hoofdlijnen.

De basisgedacht bij Lightning is dat we niet iedere transactie op de blockchain hoeven te registreren. Vergelijk het met een tijdelijke rekening bij de kroeg, waar drankjes worden aangestreept en pas aan het eind van de avond wordt afgerekend. Iedere drankje afzonderlijk betalen is tijdrovend en omslachtig. Voor bitcoin geldt min of meer hetzelfde: iedere koffie of ieder biertje registreren op de blockchain en die data verspreiden over duizenden computers over de wereld is niet schaalbaar, noch goed voor jouw privacy.

Het Lighting Netwerk heeft het potentieel om bitcoin op vele fronten te verbeteren:

\begin{itemize}
    \item Vrijwel ongelimiteerd transactievolume: honderdduizenden micro-transacties uitvoeren en eenmalig verrekenen op de blockchain. 
    
    \item Directe betaling: Wachten op confirmatie op de blokchain is niet langer nodig. 
    
    \item Extreem lage transactievergoedingen: geschikt voor micro-betalingen zoals de betaling van enkele centen om een blog te lezen. 
    
    \item Betere privacy: slechts de deelnemende partijen aan de transactie hebben er kennis van, in tegenstelling tot on-chain betalingen die met de hele wereld gedeeld worden.
    
\end{itemize}

Lightning maakt gebruik van betalingskanalen. Dit zijn on-chain bitcointransacties waarbij een hoeveelheid bitcoin wordt vastgezet en beschikbaar gemaakt binnen het Lightning Netwerk, voor directe, bijna kosteloze transacties. Het Lightning Netwerk is in de beginfase maar nu al veelbelovend. Zie als voorbeeld de website \href{https://yalls.org}{yalls.org} waar je via Lightning kunt betalen om artikelen te lezen.  

\section{Bitcoin in de ruimte}

Bitcoin is uitstekend bestand tegen censuur omdat het bestand is tegen aanvallen (je kan je eigendom in je hoofd bewaren), en bestand tegen censuur omdat er maar één eerlijke miner op het netwerk nodig is om jouw transacties uit te voeren (en je kunt zelf minen). 

Desalniettemin, aangezien bitcoin via internet wordt verstuurd is het vatbaar voor censuur op netwerkniveau. Bijvoorbeeld doordat autoritaire regimes die bitcoin willen onderdrukken, kunnen proberen om te verhinderen dat bitcoindata hun land binnenkomt en verlaat via het internet.

Het Blockstream Satellite-netwerk is een eerste poging om het netwerk uit te breiden en te beveiligen tegen censuur op staatsniveau, evenals een poging om afgelegen gebieden te bereiken die mogelijk geen verbinding met het internet hebben. Dit satellietsysteem maakt het voor iedereen met een schotel en relatief goedkope apparatuur, mogelijk om verbinding te maken en de bitcoin-blockchain te downloaden. Binnenkort wordt bidirectionele communicatie ook mogelijk. Er zijn nu ook inspanningen zoals TxTenna, die bouwen aan \textit{off-the-grid mesh-netwerken}. Als zo'n setup gecombineerd zou worden met een satellietverbinding, zou het bijna niet te stoppen zijn.

\backmatter
