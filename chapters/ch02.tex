\chapter{Inleiding}

\vspace{-1\baselineskip}
Veel mensen vormen vrij snel een mening over bitcoin als ze er voor het eerst over horen, nog voordat ze de moeite nemen om het daadwerkelijk te begrijpen. Dit wordt ook bemoeilijkt doordat ze door een zee van (verkeerde) informatie moeten ploegen om te doorgronden wat bitcoin is en hoe het functioneert. Tot drie jaar gelden, was ik ook zo iemand wiens mening gebaseerd was op onvoldoende kennis.

Waarom heb ik dit boek geschreven? De afgelopen twintig jaar heb ik me toegewijd aan het opzetten van technische startups. Dagelijks verdiep ik me in nieuwe technologieën en ik heb er een behoorlijke handigheid in ontwikkeld om uit te vogelen hoe dingen in elkaar steken. Toch heeft het vijf jaar geduurd sinds mijn eerste kennismaking met bitcoin, voordat ik besloot me er serieus in te verdiepen en het volledig te begrijpen. Ik heb sterk het vermoeden dat ik niet de enige ben die wel een steuntje in de rug kan gebruiken om deze potentieel wereldveranderende vernieuwing beter te begrijpen.

Ik hoorde in 2011 via \textit{slashdot.org}, een nieuwssite voor technologie-enthousiasten, voor het eerst over bitcoin. Destijds was de prijs opgelopen tot een enorme piek van 30 dollar per bitcoin. Mijn kennis beperkte zich tot het feit dat een groep internetgebruikers probeerden een \textit{peer-to-peer} betaalsysteem op te zetten. Zonder precies te begrijpen wat het was, hoe het functioneert en terwijl ik geen enkele ervaring had met investeringen en marktcycli, besloot ik toch om wat geld te investeren voor het geval het iets groots zou worden. Hiervoor moest ik de nogal amateuristisch ogende website van \textit{Mt. Gox} gebruiken. Dit platform voor het omwisselen van dollars naar bitcoin bleek achteraf niet competent te zijn.

Langzaam maar zeker zag ik mijn investering tot bijna niks verschrompelen, naarmate de prijs zakte van 30 dollar naar 2 dollar. Op een bepaald moment heb ik het helemaal uit mijn gedachten gezet en ben ik me weer gaan richten op mijn leven, de startups. Ik heb geen flauw idee wat er met die bitcoins is gebeurd. Vermoedelijk heb ik de sleutels ergens opgeslagen op een oude laptop die nu waarschijnlijk op de vuilnisbelt ligt.

In 2013 kwam bitcoin opnieuw op mijn pad. Ditmaal klonk het nieuws in de media een stuk luider en verliep het aankoopproces aanzienlijk soepeler. Er waren nu apps zoals \textit{Coinbase}, die een betrouwbare indruk maakten. Dit was een significante vooruitgang vergeleken met de tijd van \textit{Mt. Gox}. Ik kreeg sterk het idee dat bitcoin wel eens een succes kon worden.

Hopend dat dit het geval was en opnieuw zonder enige voorkennis, kocht ik op het hoogtepunt (rond de \$1000 per bitcoin) en zag mijn investering opnieuw instorten, ditmaal naar een waarde van slechts \$200 per bitcoin. Ik besloot dat het niet de moeite waard was om de bitcoins te verkopen en liet de situatie dus voor wat het was. Ik ging gewoon verder met mijn leven, en richtte me op mijn volgende onderneming; \textit{Reverb.com}.

In de vier jaar die volgden, maakte \textit{Reverb} een sterke groei door. Het groeide uit tot hét platform voor muzikanten. Ik maakte een verschil in de wereld door muziek met mensen te verbinden. Als CTO van een snelgroeiend en spannend technologiebedrijf, werkte ik aan iets waar ik een passie voor had. Ik had simpelweg geen tijd voor vreemde internetvaluta.

Met enige schaamte moet ik bekennen dat ik pas in de zomer van 2016 voor het eerst een video van \href{https://www.youtube.com/channel/UCJWCJCWOxBYSi5DhCieLOLQ}{\textbf{Andreas Antonopoulos}} heb bekeken. Dit zette me aan het denken over vragen zoals; waar komt bitcoin vandaan? Wie beheert het? Hoe werkt het? Wat is mining en welke impact heeft dat op de wereld? Ik raakte er steeds meer in verdiept. Anderhalf jaar lang verslond ik alles wat los en vast zat over het onderwerp. Ik luisterde urenlang naar podcasts en keek elke video over bitcoin die ik kon vinden.

En toen eindelijk, begin 2018, net nadat de koers van bitcoin een nieuwe recordhoogte had bereikt rond \$20.000 per bitcoin, besloot ik om Reverb achter me te laten en me volledig op bitcoin te richten. Waarom ik mijn succesvolle startup inruilde voor bitcoin? Ik ben ervan overtuigd dat de uitvinding van iets baanbrekends als bitcoin slechts één keer in een mensenleven voorkomt. Wellicht is het zelfs nog zeldzamer.

Als bitcoin slaagt, kan het net zo belangrijk blijken als de uitvinding van de drukpers (decentralisatie van de productie van informatie), het internet (decentralisatie van data en communicatie) en trias politicas (decentralisatie van de overheid). Mijn hoop is dat, door het begrijpen van het functioneren van bitcoin, je zult inzien hoe het de wereld kan verbeteren. Bitcoin zal de productie en consumptie van geld decentraliseren; het is het instrument dat ons als mensheid in staat stelt om ontwikkelingen te realiseren op een schaal die we nog nooit eerder voor mogelijk hielden.

In de media wordt er dikwijls eenzijdig gerapporteerd over de koers van bitcoin. Het ene moment wordt de indruk gewekt dat deze naar een miljoen dollar zal stijgen, het volgende moment wordt beweerd dat hij in een neerwaartse spiraal zit die pas stopt wanneer bitcoin volkomen waardeloos is. Daarnaast circuleren er verhalen dat bitcoin zoveel energie verbruikt dat het binnen tien jaar de aarde zou kunnen vernietigen. Natuurlijk is dit onjuist en ik hoop dat je dit zult inzien zodra je begrijpt hoe het echt werkt. Tevens zul je dan ook begrijpen waarom koers-zeepbellen feitelijk één van de minst interessante aspecten van bitcoin zijn.

In dit boek heb ik niet de intentie om de economische aspecten van bitcoin en gedegen geld te verklaren, hoewel we deze concepten wel kort zullen bespreken. Mijn doel is ook niet om bitcoin te beschrijven vanuit een investeringsperspectief of om je te overtuigen dat iedereen een stukje bitcoin zou moeten bezitten. Voor diegenen die daarin geïnteresseerd zijn, raad ik aan het boek \textit{De Bitcoin Standaard} van Saifedean Ammous te lezen, als je dat nog niet hebt gedaan.

Ook zal dit boek niet uitwijden over de computercode, en computerkennis is niet noodzakelijk om dit boek te begrijpen. Als je naar bitcoin wilt kijken vanuit dat perspectief, raad ik je Antonopoulos' \textit{Mastering Bitcoin} en Jimmy Songs \textit{Programming Bitcoin} aan.

Voor mij draait het allemaal om het begrijpen hoe verschillende factoren samenwerken om bitcoin te laten functioneren. Door middel van dit boek hoop ik deze inzichten met je te kunnen delen. Mijn doel is om je denkwijze te prikkelen en je, verrijkt met basiskennis over informatica, economie en speltheorie, inzicht te geven in hoe bitcoin een van de meest fascinerende en significante uitvindingen van deze tijd is. Als je eenmaal begrijpt hoe bitcoin werkt, hoop ik dat jij, net als ik, zult onderkennen dat het veel meer biedt dan op het eerste gezicht lijkt, en het een enorme impact kan hebben op toekomstige generaties wereldwijd.

In dit boek gaan we diep in op bitcoin en toon ik je stapsgewijs hoe alles in elkaar past. Mijn hoop is dat je voldoende inzicht vergaart om vervolgens je reis door het konijnenhol voort te zetten. Laten we van start gaan! 

\clearpage


