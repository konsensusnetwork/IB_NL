\chapter{Inleiding}

Veel mensen die voor het eerst over bitcoin horen zijn al snel geneigd om een mening te vormen voor ze een poging doen om het te begrijpen. Dat wordt bemoeilijkt doordat je door een flinke laag (mis)informatie heen moet om te begrijpen wat bitcoin is en hoe het werkt. Tot drie jaar geleden, behoorde ik ook tot een van de mensen met een mening op basis van onvoldoende kennis.

Waarom schrijf ik dit boek? De laatste twintig jaar heb ik mij gericht op het opstarten van technische start-ups. Elke dag stort ik mij in een nieuwe technologie en ik ben vrij goed geworden in het uitdokteren hoe iets werkt. Desondanks duurde het vijf jaar sinds ik voor het eerst over bitcoin hoorde, voordat ik besloot om er goed voor te gaan zitten om het te begrijpen. Ik heb het gevoel dat ik niet de enige ben die een beetje hulp kan gebruiken om deze mogelijk wereldveranderende innovatie beter te begrijpen.

Ik hoorde voor het eerst over bitcoin in 2011 van \mbox{slashdot.org}, een nieuwssite voor nerds. In die tijd was de prijs gestegen tot de enorme piek van \$30 dollar per bitcoin. Alles wat ik erover wist was dat sommige mensen op het internet probeerden om een peer-to-peer betaalsysteem op te starten. Niet wetende wat het precies was, hoe het werkte en terwijl ik niks wist over investeren en marktcycli, besloot ik toch wat geld in te leggen voor het geval het iets belangrijks zou worden. Ik moest de verschrikkelijk uitziende website van Mt. Gox gebruiken om dat te doen. Dit dollar-naar-bitcoinhandelsplatform bleek later onbekwaam.

Langzaam zag ik mijn investering krimpen tot nagenoeg niets, terwijl de prijs daalde van \$30 naar \$2. Op een gegeven moment vergat ik het volledig en ging ik verder met mijn leven, de start-ups. Ik weet niet eens wat er met die bitcoins gebeurd is. Ik denk dat ik de sleutels ergens heb opgeslagen op een oude laptop die inmiddels op de vuilnisbelt ligt.

In 2013 hoorde ik weer over bitcoin. Dit keer was het geluid in de media luider en de aanschaf ging een stuk soepeler. Er waren apps zoals Coinbase, die er legitiem uitzagen. Dit was een duidelijke vooruitgang op het tijdperk van Mt. Gox. Ik kreeg sterk het gevoel dat bitcoin weleens zou kunnen slagen.

In het geval dit zo was en weer zonder enige kennis, kocht ik weer op de piek (rond \$1000 per bitcoin) en zag mijn investering weer instorten, maar nu naar een koers van \$200 per bitcoin. Dit keer besloot ik dat het de moeite niet waard was om de bitcoins te verkopen en dus besloot ik het zo te laten. Ik ging verder en keek er niet meer naar om, omdat ik me op mijn volgende start-up richtte; Reverb.com. 

In de daaropvolgende vier jaren groeide Reverb sterk. Het werd dé website voor muzikanten. Ik betekende iets voor de wereld doordat ik muziek en mensen verbond. Ik was hoofd Technologie (CTO) van een snel groeiend en opwindend techbedrijf. Ik deed iets waar ik gepassioneerd over was en ik had geen tijd voor vreemd internetgeld.

Ik voel me beschaamd om te vertellen dat het pas in de zomer van 2016 was, dat ik voor het eerst een video van \href{https://www.youtube.com/channel/UCJWCJCWOxBYSi5DhCieLOLQ}{\textbf{Andreas Antonopoulos}} bekeek. Daardoor begon ik mezelf zaken af te vragen als; waar komen bitcoins vandaan? Wie beheert het? Hoe werkt het? Wat is mining en welke impact zal het hebben op de wereld? Ik begon me te verdiepen. Anderhalf jaar lang las ik alles waar mijn oog op viel, luisterde ik uren podcasts en keek ik elke video over bitcoin die ik tegenkwam.

En toen eindelijk, begin 2018, net nadat de koers van bitcoin een nieuwe recordhoogte had bereikt rond \$20.000 per bitcoin, besloot ik om Reverb achter me te laten en me volledig op bitcoin te richten. Waarom ik mijn succesvolle start-up verliet voor bitcoin? Omdat ik geloof dat een uitvinding van iets als bitcoin, maar een keer in een mensenleven zich voordoet. En misschien zelfs nog minder vaak.

Als bitcoin slaagt, kan het net zo belangrijk blijken als de uitvinding van de drukpers (decentralisatie van de productie van informatie), het internet (decentralisatie van data en communicatie) en trias politicas (decentralisatie van de overheid). Ik hoop dat door te begrijpen hoe bitcoin werkt, je zal begrijpen hoe het de wereld ten goede kan komen. Bitcoin zal de productie en consumptie van geld decentraliseren; dat is het middel waarmee de mensheid tot nieuwe ontwikkelingen kan komen op een schaal die voorheen ondenkbaar was.

In de media gaat het vaak alleen over de koers van bitcoin. Het ene moment wordt de suggestie gewekt dat hij naar een miljoen dollar gaat, het andere moment zit hij in een neerwaartse spiraal die pas zal stoppen als bitcoin waardeloos is geworden. En anders zijn er wel verhalen dat bitcoin zoveel energie gebruikt dat het de aarde binnen tien jaar zal vernietigen. Dit is natuurlijk fout en ik hoop dat je dat zal begrijpen zodra je leert hoe het werkt. Je zal ook begrijpen waarom koers-bubbels de minst interessante dingen zijn wat betreft bitcoin. 

Met dit boek probeer ik niet de economie van bitcoin en gedegen geld uit te leggen, hoewel we die concepten wel kort zullen aanraken. Ik ga bitcoin ook niet beschrijven vanuit een investeringsstandpunt of je overtuigen dat iedereen een beetje bitcoin zou moeten hebben. Ik raad iedereen aan om \textit{De Bitcoin Standaard} van Saifedean Ammous te lezen als je dat nog niet hebt gedaan.\footnote{Nvdr: ook dit boek is inmiddels in het Nederlands vertaald en verkrijgbaar via Konsensus Netwerk.\footnote{\url{https://konsensus.network/shop/}}}

Ook zal dit boek niet uitwijden over de computercode, en computerkennis is niet noodzakelijk om dit boek te begrijpen. Als je naar bitcoin wilt kijken vanuit dat perspectief, raad ik je Antonopoulos' \textit{Mastering Bitcoin} en Jimmy Songs \textit{Programming Bitcoin} aan.

Voor mij gaat het om het begrijpen hoe alle dingen samenkomen waardoor bitcoin werkt. Met dit boek hoop ik die kennis met je te kunnen delen. Mijn doel is om je hersenen een beetje te kietelen en om je met een vleugje computerkennis, economisch en speltheoretisch inzicht te geven hoe bitcoin een van de meest interessante en belangrijkste uitvindingen is van deze tijd. Als je begrijpt hoe bitcoin werkt hoop ik dat jij, net als ik, zal inzien dat bitcoin veel meer is dan het op het eerste gezicht lijkt en dat het een geweldige impact zal hebben op de volgende generaties van deze wereld.

In dit boek nemen we bitcoin onder de loep en laat ik je stap voor stap zien hoe alles samenkomt. Ik hoop dat je genoeg kennis opdoet om daarna je reis \textit{down the rabbit hole} te vervolgen. Laten we beginnen! 

\clearpage


